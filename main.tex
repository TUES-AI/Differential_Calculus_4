% !TEX program = pdflatex
\documentclass[11pt,a4paper]{article}

% --- Български (pdfLaTeX) ---
\usepackage[T2A]{fontenc}
\usepackage[utf8]{inputenc}
\usepackage[bulgarian]{babel}

% --- Математика и форматиране ---
\usepackage{amsmath, amssymb, mathtools}
\usepackage{geometry}
\usepackage{xcolor}
\usepackage{booktabs}
\usepackage{microtype}
\usepackage{enumitem}

% --- Графики (векторни) ---
\usepackage{tikz}
\usepackage{pgfplots}
\pgfplotsset{compat=1.18}

\geometry{margin=2.2cm}

% --- Цветове: основа (червено) и аргумент (зелено) ---
\definecolor{BaseRed}{RGB}{200,40,40}
\definecolor{ArgGreen}{RGB}{20,140,60}

\newcommand{\base}[1]{\textcolor{BaseRed}{#1}}
\newcommand{\argu}[1]{\textcolor{ArgGreen}{#1}}

% \logc{a}{x} -> \log_a(x) с оцветена основа и аргумент
\newcommand{\logc}[2]{\log_{\base{#1}}\!\left(\argu{#2}\right)}
\newcommand{\lnc}[1]{\ln\!\left(\argu{#1}\right)}

% Оцветена степен (основа в червено)
\newcommand{\powc}[2]{\base{#1}^{#2}}

% Инверсни тъждества (за “запомняне”)
\newcommand{\invexp}[2]{\powc{#1}{\logc{#1}{#2}} = \argu{#2}}
\newcommand{\invlog}[2]{\logc{#1}{\powc{#1}{#2}} = #2}

% Таблично оформление
\renewcommand{\arraystretch}{1.6}
\setlength{\tabcolsep}{10pt}

% Задачи
\newlist{zadachi}{enumerate}{1}
\setlist[zadachi]{label=\textbf{Задача \arabic*.}, leftmargin=*, itemsep=0.45\baselineskip}

\title{Експонента, логаритми, граници и производни\\\large (обобщение + задачи с решения)}
\author{}
\date{}

\begin{document}
\maketitle

\section{Цели на урока}
В този урок обединяваме:
\begin{itemize}
  \item \textbf{граници} (основни правила и стандартни граници);
  \item \textbf{производни} и \textbf{правила за диференциране};
  \item \textbf{експоненциална функция} \(a^x\) и специалната основа \(e\);
  \item \textbf{логаритми} \(\log_a x\) и естествен логаритъм \(\ln x\);
  \item набор от задачи, които покриват всички теми, с подробни решения.
\end{itemize}

\section{Бързо обобщение: граници}
\subsection{Основни правила}
Ако съществуват \(\lim_{x\to x_0} f(x)\) и \(\lim_{x\to x_0} g(x)\), то:
\[
\lim_{x\to x_0}(f+g)=\lim_{x\to x_0}f+\lim_{x\to x_0}g,\qquad
\lim_{x\to x_0}(fg)=\left(\lim_{x\to x_0}f\right)\left(\lim_{x\to x_0}g\right),
\]
\[
\lim_{x\to x_0}\frac{f}{g}=\frac{\lim_{x\to x_0}f}{\lim_{x\to x_0}g}
\quad \text{ако } \lim_{x\to x_0}g\neq 0.
\]

\subsection{Ключови стандартни граници}
\[
\lim_{x\to 0}\frac{\sin x}{x}=1,\qquad
\lim_{x\to 0}\frac{e^x-1}{x}=1,\qquad
\lim_{x\to 0}\frac{\ln(1+x)}{x}=1.
\]

\section{Бързо обобщение: производни}
\subsection{Определение}
\[
f'(x_0)=\lim_{h\to 0}\frac{f(x_0+h)-f(x_0)}{h},
\]
ако границата съществува.

\subsection{Таблица за производни и правила за диференциране (както на слайда)}
\noindent
\begin{minipage}[t]{0.48\textwidth}
\centering
\textbf{Таблица за производни}

\vspace{0.3\baselineskip}
\small
\begin{tabular}{|l|l|}
\hline
Функция $f(x)$ & Производна $f'(x)$ \\
\hline
$c$ (константа) & $0$ \\
\hline
$x^n$ ($n$ константа) & $n x^{n-1}$ \\
\hline
$\sqrt{x}$ & $\dfrac{1}{2\sqrt{x}}$ \\
\hline
$\dfrac{1}{x}$ & $-\dfrac{1}{x^2}$ \\
\hline
$\sin x$ & $\cos x$ \\
\hline
$\cos x$ & $-\sin x$ \\
\hline
$\tan x$ & $\dfrac{1}{\cos^2 x}$ \\
\hline
$\ln x$ & $\dfrac{1}{x}$ \\
\hline
$e^x$ & $e^x$ \\
\hline
$a^x$ ($a>0, a\neq 1$) & $a^x\ln a$ \\
\hline
$\log_a x$ ($a>0,a\neq 1$) & $\dfrac{1}{x\ln a}$ \\
\hline
\end{tabular}
\end{minipage}\hfill
\begin{minipage}[t]{0.48\textwidth}
\centering
\textbf{Правила за диференциране}

\vspace{0.3\baselineskip}
\small
\begin{tabular}{|l|l|}
\hline
Правило & Формула \\
\hline
Константен множител & $(c f(x))' = c f'(x)$ \\
\hline
Сума и разлика & $(f \pm g)' = f' \pm g'$ \\
\hline
Произведение & $(fg)' = f'g + fg'$ \\
\hline
Частно & $\left(\dfrac{f}{g}\right)' = \dfrac{f'g - fg'}{g^2}\ (g\neq 0)$ \\
\hline
Верижно правило & $(f(g(x)))' = f'(g(x))\cdot g'(x)$ \\
\hline
\end{tabular}
\end{minipage}

\section{Експонента}
\subsection{Определение и базови свойства}
За \(a>0\), \(a\neq 1\) експоненциалната функция е \(f(x)=a^x\).
Полезни свойства:
\[
a^{x+y}=a^x a^y,\qquad
a^{x-y}=\frac{a^x}{a^y},\qquad
(a^x)^y=a^{xy}.
\]

\subsection{Графика и монотонност (върната графика от оригинала)}
Ако \(a>1\), \(a^x\) е растяща; ако \(0<a<1\), \(a^x\) е намаляваща.

\begin{figure}[h]
\centering
\begin{tikzpicture}
\begin{axis}[
  axis lines=middle,
  xlabel={$x$}, ylabel={$y$},
  xmin=-3, xmax=3,
  ymin=0, ymax=8.5,
  xtick={-3,-2.5,-2,-1.5,-1,-0.5,0.5,1,1.5,2,2.5,3},
  ytick={2,4,6,8},
  legend style={at={(0.02,0.98)},anchor=north west,draw=black,fill=white},
  width=0.88\textwidth,
  height=0.32\textheight
]
\addplot[domain=-3:3,samples=240,thick,blue] {pow(2,x)};
\addlegendentry{$2^x$ (растяща)}
\addplot[domain=-3:3,samples=240,thick,red] {pow(0.5,x)};
\addlegendentry{$(1/2)^x$ (намаляваща)}
\addplot[domain=-3:3,samples=2,dashed,gray] {1};
\end{axis}
\end{tikzpicture}
\caption{Сравнение на \(2^x\) и \((1/2)^x\) (както в оригиналния материал).}
\end{figure}

\subsection{Специалната основа \(e\)}
Класическият предел:
\[
e=\lim_{n\to\infty}\left(1+\frac{1}{n}\right)^n \approx 2.71828\ldots
\]
Уникално свойство:
\[
\frac{d}{dx}e^x=e^x.
\]

\section{Логаритми}
\subsection{Дефиниция и инверсия}
За \(a>0\), \(a\neq 1\), \(x>0\):
\[
\log_a x = y \quad \Longleftrightarrow \quad a^y=x.
\]
Естествен логаритъм: \(\ln x=\log_e x\).

\paragraph{Оцветена “инверсия” (както поиска).}
\[
\invexp{a}{x},
\qquad
\invlog{a}{x}.
\]

\subsection{Правила за логаритми (в табличен вид)}
\begin{center}
\textbf{Правила за логаритми (за \(x>0\), \(y>0\), \(a>0\), \(a\neq 1\))}
\vspace{0.3\baselineskip}

\small
\begin{tabular}{|p{0.36\textwidth}|p{0.56\textwidth}|}
\hline
\textbf{Правило} & \textbf{Формула} \\
\hline
Произведение & $\log_a(xy)=\log_a x+\log_a y$ \\
\hline
Частно & $\log_a\!\left(\dfrac{x}{y}\right)=\log_a x-\log_a y$ \\
\hline
Степен & $\log_a(x^r)=r\,\log_a x\quad (r\in\mathbb{R})$ \\
\hline
Смяна на основата & $\displaystyle \log_a x=\frac{\ln x}{\ln a}$ \\
\hline
Инверсия I & $\powc{a}{\logc{a}{x}}=\argu{x}$ \\
\hline
Инверсия II & $\logc{a}{\powc{a}{x}}=x$ \\
\hline
\end{tabular}
\end{center}

\subsection{Графики на \(\ln x\) и \(\log_2 x\) (върната графика от оригинала)}
\begin{figure}[h]
\centering
\begin{tikzpicture}
\begin{axis}[
  axis lines=middle,
  xlabel={$x$}, ylabel={$y$},
  xmin=0.1, xmax=6.5,
  ymin=-2.2, ymax=2.4,
  xtick={0,1,2,3,4,5,6},
  ytick={-2,-1,0,1,2},
  legend style={at={(0.10,0.25)},anchor=south west,draw=black,fill=white},
  width=0.88\textwidth,
  height=0.30\textheight
]
\addplot[domain=0.1:6.5,samples=260,thick,blue] {ln(x)};
\addlegendentry{$\ln x$}
\addplot[domain=0.1:6.5,samples=260,thick,red] {ln(x)/ln(2)};
\addlegendentry{$\log_2 x$}
\end{axis}
\end{tikzpicture}
\caption{Сравнение на \(\ln x\) и \(\log_2 x\) (както в оригиналния материал).}
\end{figure}

\subsection{Производни на логаритми}
\[
\frac{d}{dx}\ln x=\frac{1}{x}\ (x>0),
\qquad
\frac{d}{dx}\log_a x=\frac{1}{x\ln a}\ (x>0).
\]

\section{Съчетаване на темите}
\subsection{Типични неопределености}
\[
\frac{0}{0},\ \frac{\infty}{\infty},\ 0\cdot\infty,\ \infty-\infty,\ 1^\infty,\ 0^0,\ \infty^0.
\]
\subsection{Методи}
\begin{itemize}
  \item алгебрични преобразувания (изнасяне, рационализиране);
  \item замяна \(t=g(x)\);
  \item използване на стандартни граници;
  \item Лопитал (ако е изучавано) при \(\frac00\) или \(\frac{\infty}{\infty}\).
\end{itemize}

\section{Задачи}
\begin{zadachi}
  \item Изчислете: \(\ln(e^3)\).
  \item Изчислете: \(e^{\ln 7}\).
  \item Опростете: \(\ln(e^{2x})\).
  \item Опростете: \(e^{\ln(5x)}\) за \(x>0\).
  \item Изчислете: \(\ln\!\left(\dfrac{e^4}{e}\right)\).
  \item Решете уравнението: \(e^{2x}=9\).
  \item Решете уравнението: \(\ln x=-2\).
  \item Решете уравнението: \(\ln(x-1)=\ln(2x+3)-\ln 5\).
  \item Решете уравнението: \(\ln x+\ln(x-2)=\ln 3\).
  \item Решете уравнението: \(e^x+e^{-x}=4\).
  \item Изчислете производната: \(y=e^{3x}\).
  \item Изчислете производната: \(y=\ln(3x-1)\).
  \item Изчислете производната: \(y=(x^2+1)e^x\).
  \item Изчислете производната: \(y=\ln\!\left(\dfrac{x^2+1}{e^x}\right)\) (за \(x>0\)).
  \item Намерете уравнението на допирателната към \(y=\ln x\) в точка \(x=1\).
  \item Намерете \(\displaystyle \lim_{x\to 0}\frac{e^x-1}{x}\).
  \item Намерете \(\displaystyle \lim_{x\to 0}\frac{\ln(1+x)}{x}\).
  \item (едно неравенство) Решете: \(\ln x\le 2\).
  \item Изразете чрез естествен логаритъм: \(\log_2 e\).
  \item Намерете всички \(x\), за които е вярно: \(\log_x e=2\).
  \item Решете уравнението: \(\ln(x+1)=2\).
  \item Решете уравнението: \(\ln(x-1)+\ln(x+1)=\ln 8\).
  \item Решете уравнението: \(\ln(e^x-1)=0\).
  \item Решете уравнението: \(\log_3(x+1)-\log_3(x-2)=1\).
  \item Решете уравнението: \(\log_2 x=\log_2 5-3\).
\end{zadachi}

\section{Подробни решения (стъпка по стъпка)}

\subsection*{Решение на задача 1}
\[
\ln(e^3)=3,
\]
защото \(\ln(e^t)=t\) за всеки реален \(t\).

\subsection*{Решение на задача 2}
\[
e^{\ln 7}=7,
\]
защото \(e^{\ln x}=x\) за \(x>0\).

\subsection*{Решение на задача 3}
За всеки реален \(x\) имаме \(e^{2x}>0\), следователно:
\[
\ln(e^{2x})=2x.
\]

\subsection*{Решение на задача 4}
Условие за дефинираност: \(5x>0 \Rightarrow x>0\). Тогава:
\[
e^{\ln(5x)}=5x.
\]

\subsection*{Решение на задача 5}
\[
\ln\!\left(\frac{e^4}{e}\right)=\ln(e^{4-1})=\ln(e^3)=3.
\]

\subsection*{Решение на задача 6}
\[
e^{2x}=9
\ \Longleftrightarrow\ 
2x=\ln 9
\ \Longleftrightarrow\ 
x=\frac{\ln 9}{2}.
\]
(Еквивалентно: \(\ln 9=2\ln 3\Rightarrow x=\ln 3\).)

\subsection*{Решение на задача 7}
\[
\ln x=-2
\ \Longleftrightarrow\ 
x=e^{-2}.
\]

\subsection*{Решение на задача 8}
\textbf{Област:} \(x-1>0 \Rightarrow x>1\).

\[
\ln(x-1)=\ln(2x+3)-\ln 5
= \ln\!\left(\frac{2x+3}{5}\right).
\]
Следователно (понеже \(\ln\) е инективна върху \((0,\infty)\)):
\[
x-1=\frac{2x+3}{5}
\ \Longleftrightarrow\ 
5x-5=2x+3
\ \Longleftrightarrow\ 
3x=8
\ \Longleftrightarrow\ 
x=\frac{8}{3}.
\]
Проверка: \(\frac{8}{3}>1\) — допустимо.

\subsection*{Решение на задача 9}
\textbf{Област:} \(x>0\) и \(x-2>0 \Rightarrow x>2\).

\[
\ln x+\ln(x-2)=\ln(x(x-2))=\ln 3
\ \Longleftrightarrow\ 
x(x-2)=3.
\]
\[
x^2-2x-3=0
\ \Longleftrightarrow\ 
(x-3)(x+1)=0.
\]
От областта \(x>2\) остава:
\[
x=3.
\]

\subsection*{Решение на задача 10}
Нека \(t=e^x\), тогава \(t>0\) и \(e^{-x}=\frac{1}{t}\). Получаваме:
\[
t+\frac{1}{t}=4
\ \Longleftrightarrow\ 
t^2-4t+1=0.
\]
\[
t=\frac{4\pm\sqrt{16-4}}{2}=2\pm\sqrt{3}.
\]
И двете стойности са положителни, следователно:
\[
e^x=2+\sqrt{3}\ \Rightarrow\ x=\ln(2+\sqrt{3}),
\]
\[
e^x=2-\sqrt{3}\ \Rightarrow\ x=\ln(2-\sqrt{3}).
\]

\subsection*{Решение на задача 11}
\[
y=e^{3x}.
\]
По верижно правило:
\[
y' = e^{3x}\cdot (3)=3e^{3x}.
\]

\subsection*{Решение на задача 12}
\[
y=\ln(3x-1).
\]
По верижно правило:
\[
y'=\frac{1}{3x-1}\cdot 3=\frac{3}{3x-1}.
\]

\subsection*{Решение на задача 13}
\[
y=(x^2+1)e^x.
\]
По правило за произведение:
\[
y'=(x^2+1)'\,e^x+(x^2+1)(e^x)'
= (2x)e^x+(x^2+1)e^x.
\]
\[
y'=e^x(2x+x^2+1)=e^x(x^2+2x+1)=e^x(x+1)^2.
\]

\subsection*{Решение на задача 14}
\[
y=\ln\!\left(\frac{x^2+1}{e^x}\right).
\]
Използваме \(\ln\left(\frac{A}{B}\right)=\ln A-\ln B\):
\[
y=\ln(x^2+1)-\ln(e^x)=\ln(x^2+1)-x.
\]
Диференцираме:
\[
y'=\frac{2x}{x^2+1}-1.
\]

\subsection*{Решение на задача 15}
Функция: \(y=\ln x\). Тогава \(y'=\frac{1}{x}\).

В точка \(x_0=1\):
\[
y(1)=\ln 1=0,\qquad y'(1)=1.
\]
Уравнение на допирателната:
\[
y-y(1)=y'(1)(x-1)\ \Longleftrightarrow\ y= x-1.
\]

\subsection*{Решение на задача 16}
Това е стандартната граница:
\[
\lim_{x\to 0}\frac{e^x-1}{x}=1.
\]

\subsection*{Решение на задача 17}
Това е стандартната граница:
\[
\lim_{x\to 0}\frac{\ln(1+x)}{x}=1.
\]

\subsection*{Решение на задача 18}
\textbf{Област:} \(x>0\). Понеже \(\ln x\) е растяща функция:
\[
\ln x\le 2
\ \Longleftrightarrow\ 
x\le e^2.
\]
Следователно:
\[
0<x\le e^2.
\]

\subsection*{Решение на задача 19}
Смяна на основата:
\[
\log_2 e=\frac{\ln e}{\ln 2}=\frac{1}{\ln 2}.
\]

\subsection*{Решение на задача 20}
\[
\log_x e=2.
\]
По дефиниция на логаритъм това е еквивалентно на:
\[
x^2=e.
\]
\textbf{Област:} \(x>0\), \(x\neq 1\). Положителното решение е:
\[
x=\sqrt{e}.
\]

\subsection*{Решение на задача 21}
\textbf{Област:} \(x+1>0\Rightarrow x>-1\).
\[
\ln(x+1)=2
\ \Longleftrightarrow\ 
x+1=e^2
\ \Longleftrightarrow\ 
x=e^2-1.
\]

\subsection*{Решение на задача 22}
\textbf{Област:} \(x-1>0\) и \(x+1>0\Rightarrow x>1\).
\[
\ln(x-1)+\ln(x+1)=\ln\big((x-1)(x+1)\big)=\ln(x^2-1).
\]
Така:
\[
\ln(x^2-1)=\ln 8
\ \Longleftrightarrow\ 
x^2-1=8
\ \Longleftrightarrow\ 
x^2=9.
\]
От областта \(x>1\) остава:
\[
x=3.
\]

\subsection*{Решение на задача 23}
\textbf{Област:} \(e^x-1>0\Rightarrow x>0\).
\[
\ln(e^x-1)=0
\ \Longleftrightarrow\ 
e^x-1=1
\ \Longleftrightarrow\ 
e^x=2
\ \Longleftrightarrow\ 
x=\ln 2.
\]

\subsection*{Решение на задача 24}
\textbf{Област:} \(x+1>0\) и \(x-2>0\Rightarrow x>2\).
\[
\log_3(x+1)-\log_3(x-2)=\log_3\!\left(\frac{x+1}{x-2}\right)=1.
\]
По дефиниция: \(\log_3(\cdot)=1 \Longleftrightarrow (\cdot)=3\). Значи:
\[
\frac{x+1}{x-2}=3
\ \Longleftrightarrow\ 
x+1=3x-6
\ \Longleftrightarrow\ 
2x=7
\ \Longleftrightarrow\ 
x=\frac{7}{2}.
\]
Проверка: \(\frac{7}{2}>2\) — допустимо.

\subsection*{Решение на задача 25}
\textbf{Област:} \(x>0\).
\[
\log_2 x=\log_2 5-3.
\]
Представяме \(3\) като \(\log_2 8\):
\[
\log_2 x=\log_2 5-\log_2 8=\log_2\!\left(\frac{5}{8}\right).
\]
Понеже \(\log_2\) е инективна:
\[
x=\frac{5}{8}.
\]

\end{document}


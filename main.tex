% !TEX program = pdflatex
\documentclass[11pt,a4paper]{article}

% --- Български език (pdfLaTeX) ---
\usepackage[T2A]{fontenc}
\usepackage[utf8]{inputenc}
\usepackage[bulgarian]{babel}

% --- Математика и форматиране ---
\usepackage{amsmath, amssymb, mathtools}
\usepackage{geometry}
\usepackage{xcolor}
\usepackage{booktabs}
\usepackage{microtype}

\geometry{margin=2.2cm}

% --- Цветове: основа (червено) и аргумент (зелено) ---
\definecolor{BaseRed}{RGB}{200,40,40}
\definecolor{ArgGreen}{RGB}{20,140,60}

\newcommand{\base}[1]{\textcolor{BaseRed}{#1}}
\newcommand{\argu}[1]{\textcolor{ArgGreen}{#1}}

% Логаритъм с оцветена основа и аргумент:
% \logc{a}{x} -> \log_{a}(x), но a е червено, x е зелено
\newcommand{\logc}[2]{\log_{\base{#1}}\!\left(\argu{#2}\right)}

% Естествен логаритъм с оцветен аргумент (по желание):
\newcommand{\lnc}[1]{\ln\!\left(\argu{#1}\right)}

% Оцветена степен: \powc{a}{x} -> a^x, но a е червено (като "основа")
\newcommand{\powc}[2]{\base{#1}^{#2}}

% Оцветени “инверсни” тъждества (удобно за правила като a^{log_a(x)}=x):
\newcommand{\invexp}[2]{\powc{#1}{\logc{#1}{#2}} = \argu{#2}}
\newcommand{\invlog}[2]{\logc{#1}{\powc{#1}{#2}} = #2}

% Малко по-“книжно” разстояние между редовете в таблици
\renewcommand{\arraystretch}{1.6}
\setlength{\tabcolsep}{10pt}

\title{Експонента, логаритми, граници и производни\\\large (обобщение + основни правила)}
\author{}
\date{}

\begin{document}
\maketitle

\section{Цели и контекст}
В този раздел обединяваме четири близко свързани теми, които постоянно се появяват и в
машинното обучение:
\begin{itemize}
  \item \textbf{граници} (за да разбираме дефиниции и “стандартни” предели);
  \item \textbf{производни} и \textbf{правила за диференциране} (за градиенти и оптимизация);
  \item \textbf{експоненциална функция} $a^x$ и специалната основа $e$ (ключова в softmax, log-sum-exp, вероятности);
  \item \textbf{логаритми} $\log_a x$ и естествен логаритъм $\ln x$ (ключови в log-loss, cross-entropy, стабилност).
\end{itemize}

\section{Бързо обобщение: граници}
\subsection{Интуитивна идея}
Записът $\lim\limits_{x\to x_0} f(x)=L$ означава:
когато $x$ е достатъчно близо до $x_0$ (без да е задължително равно на $x_0$),
стойностите на $f(x)$ стават достатъчно близки до $L$.

\subsection{Основни правила}
Ако съществуват $\lim\limits_{x\to x_0} f(x)$ и $\lim\limits_{x\to x_0} g(x)$, то:
\[
\lim_{x\to x_0}(f(x)+g(x))=\lim_{x\to x_0}f(x)+\lim_{x\to x_0}g(x),
\]
\[
\lim_{x\to x_0}(f(x)\,g(x))=
\left(\lim_{x\to x_0}f(x)\right)\left(\lim_{x\to x_0}g(x)\right),
\]
\[
\lim_{x\to x_0}\frac{f(x)}{g(x)}=
\frac{\lim_{x\to x_0}f(x)}{\lim_{x\to x_0}g(x)}
\quad\text{ако } \lim_{x\to x_0}g(x)\neq 0.
\]

\subsection{Ключови стандартни граници}
Тези предели стоят в основата на производните на $\sin x$, $e^x$ и $\ln(1+x)$:
\[
\lim_{x\to 0}\frac{\sin x}{x}=1,\qquad
\lim_{x\to 0}\frac{e^x-1}{x}=1,\qquad
\lim_{x\to 0}\frac{\ln(1+x)}{x}=1.
\]

\section{Производни: дефиниция, таблица и правила}
\subsection{Дефиниция}
Производната на $f$ в точка $x_0$ се дефинира като граница:
\[
f'(x_0)=\lim_{h\to 0}\frac{f(x_0+h)-f(x_0)}{h},
\]
ако границата съществува. Интуитивно: $f'(x_0)$ е наклонът на допирателната към графиката в точката.

\subsection{Две основни таблици (както на слайда)}
\noindent
\begin{minipage}[t]{0.48\textwidth}
\centering
\textbf{Таблица за производни}

\vspace{0.3\baselineskip}
\small
\begin{tabular}{|l|l|}
\hline
Функция $f(x)$ & Производна $f'(x)$ \\
\hline
$c$ (константа) & $0$ \\
\hline
$x^n$ ($n$ константа) & $n x^{n-1}$ \\
\hline
$\sqrt{x}$ & $\dfrac{1}{2\sqrt{x}}$ \\
\hline
$\dfrac{1}{x}$ & $-\dfrac{1}{x^2}$ \\
\hline
$\sin x$ & $\cos x$ \\
\hline
$\cos x$ & $-\sin x$ \\
\hline
$\tan x$ & $\dfrac{1}{\cos^2 x}$ \\
\hline
$\ln x$ & $\dfrac{1}{x}$ \\
\hline
$e^x$ & $e^x$ \\
\hline
$a^x$ ($a>0, a\neq 1$) & $a^x\ln a$ \\
\hline
$\log_a x$ ($a>0,a\neq 1$) & $\dfrac{1}{x\ln a}$ \\
\hline
\end{tabular}
\end{minipage}\hfill
\begin{minipage}[t]{0.48\textwidth}
\centering
\textbf{Правила за диференциране}

\vspace{0.3\baselineskip}
\small
\begin{tabular}{|l|l|}
\hline
Правило & Формула \\
\hline
Константен множител & $(c f(x))' = c f'(x)$ \\
\hline
Сума и разлика & $(f \pm g)' = f' \pm g'$ \\
\hline
Произведение & $(fg)' = f'g + fg'$ \\
\hline
Частно & $\left(\dfrac{f}{g}\right)' = \dfrac{f'g - fg'}{g^2}\quad (g\neq 0)$ \\
\hline
Верижно правило & $(f(g(x)))' = f'(g(x))\cdot g'(x)$ \\
\hline
\end{tabular}
\end{minipage}

\subsection{Верижно правило (chain rule) — защо е толкова важно}
В машинното обучение почти всяка функция е композиция:
\[
\text{loss}(\theta) = \ell\big(\underbrace{g(\theta)}_{\text{модел}}\big),
\]
затова \textbf{chain rule} е „механизмът“ зад backpropagation.

\paragraph{Пример.}
Нека $y = \ln(3x-1)$. Това е композиция: вътрешна функция $u(x)=3x-1$ и външна $f(u)=\ln u$.
Тогава:
\[
y' = \frac{1}{u(x)}\cdot u'(x)=\frac{1}{3x-1}\cdot 3=\frac{3}{3x-1}.
\]

\section{Експонента и специалната основа $e$}
\subsection{Експоненциална функция}
За $a>0$ и $a\neq 1$ дефинираме $f(x)=a^x$. Полезни правила за степени:
\[
a^{x+y}=a^x a^y,\qquad
a^{x-y}=\frac{a^x}{a^y},\qquad
(a^x)^y=a^{xy}.
\]

\subsection{Откъде идва $e$ (идея за непрекъснато олихвяване)}
Класическият начин да се “роди” $e$ е пределът:
\[
e=\lim_{n\to\infty}\left(1+\frac{1}{n}\right)^n \approx 2.71828\ldots
\]
Тази основа е специална, защото има уникалното свойство:
\[
\frac{d}{dx}e^x=e^x.
\]

\subsection{Производна на $a^x$}
Често се ползва представянето $a^x=e^{x\ln a}$, което дава:
\[
\frac{d}{dx}a^x=\frac{d}{dx}e^{x\ln a}=e^{x\ln a}\cdot \ln a = a^x\ln a.
\]

\section{Логаритми: дефиниция, инверсия и правила}
\subsection{Дефиниция и връзка с експонентата}
За $a>0$, $a\neq 1$ и $x>0$:
\[
\log_a x = y \quad\Longleftrightarrow\quad a^y=x.
\]
Естественият логаритъм е $\ln x=\log_e x$.

\subsection{“Оцветени” инверсни тъждества (за по-лесно запомняне)}
С използване на макросите по-горе:
\[
\invexp{a}{x}
\qquad\text{и}\qquad
\invlog{a}{x}.
\]
Тоест (в разписан вид):
\[
\textcolor{BaseRed}{a}^{\log_{\textcolor{BaseRed}{a}}\!\left(\textcolor{ArgGreen}{x}\right)}=\textcolor{ArgGreen}{x},
\qquad
\log_{\textcolor{BaseRed}{a}}\!\left(\textcolor{BaseRed}{a}^{x}\right)=x.
\]

\subsection{Таблица: правила за логаритми}
\begin{center}
\textbf{Правила за логаритми (за $x>0$, $y>0$, $a>0$, $a\neq 1$)}
\vspace{0.3\baselineskip}

\small
\begin{tabular}{|p{0.36\textwidth}|p{0.56\textwidth}|}
\hline
\textbf{Правило} & \textbf{Формула} \\
\hline
Произведение & $\log_a(xy)=\log_a x+\log_a y$ \\
\hline
Частно & $\log_a\!\left(\dfrac{x}{y}\right)=\log_a x-\log_a y$ \\
\hline
Степен (показател отпред) & $\log_a(x^r)=r\,\log_a x\quad (r\in\mathbb{R})$ \\
\hline
Смяна на основата & $\displaystyle \log_a x=\frac{\ln x}{\ln a}$ \\
\hline
Инверсия I & $a^{\log_a x}=x$ \\
\hline
Инверсия II & $\log_a(a^x)=x$ \\
\hline
Логаритъм от 1 & $\log_a 1 = 0$ \\
\hline
Логаритъм от основата & $\log_a a = 1$ \\
\hline
Монотонност (важно за неравенства) & ако $a>1$ — $\log_a x$ е растяща; ако $0<a<1$ — намаляваща \\
\hline
\end{tabular}
\end{center}

\subsection{Производни на логаритми}
Ключов резултат:
\[
\frac{d}{dx}\ln x=\frac{1}{x}\quad (x>0).
\]
От смяна на основата $\log_a x=\dfrac{\ln x}{\ln a}$ следва:
\[
\frac{d}{dx}\log_a x=\frac{1}{x\ln a}\quad (x>0).
\]

\section{Мини-примери (типични в ML контекст)}
\subsection{Лог-вероятност и производна}
Нека $L(x)= -\ln(x)$ (напр. негативен лог-вероятностен член). Тогава:
\[
L'(x) = -\frac{1}{x},
\]
което показва защо при много малки вероятности градиентите могат да станат големи по абсолютна стойност.

\subsection{Експонента в композиция}
Ако $y=e^{3x}$, то по chain rule:
\[
y' = e^{3x}\cdot 3 = 3e^{3x}.
\]

\end{document}
